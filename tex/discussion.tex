%#!pdflatex Naruse_Esurf_2020.tex
\section{Discussion}

\subsection{Performance of inverse model}
The performance of the inverse model for turbidity currents is evaluated using the test data set, implying that this model can accurately reconstruct the flow characteristics of the turbidity currents from the spatial distribution of the thickness and grain size of turbidites (Figs. \ref{fig:test_scatter_plot} and \ref{fig:test_histogram_deviation}). The biases in the values reconstructed from the true input parameters are also very small and thus should not pose a serious issue when the method is applied to actual field data.  

The inverse model not only reconstructed the initial conditions of turbidity currents accurately, but also the predicted time evolution of the flow behavior was sufficiently accurately and precisely. In the results of the forward model calculations using the predicted model input parameters that are relatively deviate from the true values (Table \ref{table:example_time_evolution}), the time evolution of the velocity and the thickness of the flow does not deviate significantly from the results using the true values (Fig. \ref{fig:test_example_time_evolution}).

Turbidity currents have a mechanism called the self-acceleration, which is caused by erosion and associated increase of the flow density \citep{parker1986self,Naruse2007,Sequeiros2009}. Therefore, even slight differences in the initial conditions of the flow can lead to very different results of the time evolution of the flow parameters. However, the results of this test imply that the accuracy of the inverse analysis in this study is enough to prevent to cause such a drastic change in the flow behavior.

The relationship between turbidity currents and characteristics of turbidites is nonlinear. Especially when the flow is self-accelerating, a small difference in the initial conditions can result in very different sedimentary characteristics. This means that it is easy to find the initial conditions of the flow by inverse analysis, because even if the characteristics of the deposits are very different, the initial conditions of the flow should not be so different. Thus, the inverse results in this case are expected to be robust even if there are some measurement errors in characteristics of deposits. In other words, there is a tradeoff between the robustness of the forward and inverse modeling.

This property of the inversion can be understood when we consider the opposite case. If the initial conditions of the flow are different but the characteristics of the turbidites are exactly the same, it is impossible to estimate the flow conditions from the turbidites. The inverse analysis of hydraulic conditions is possible because the depositional characteristics are sensitive to conditions of turbidity currents. The self-acceleration of turbidity flow is an extreme example of the sensitivity of turbidites to the flow initial conditions.

\subsection{Applicability to field-scale problems}
To apply this method to outcrops, the extent of the area that should be surveyed to collect data and the interval between outcrops should be determined. The tests with different sizes of sampling windows suggest that the survey region should be located more than 10 km from the proximal region (Fig. \ref{fig:training_different_number_length}). The loss function (i.e., the MSE of the estimates of the parameters) decreases as the length of the sampling window increases, and the best result is obtained at the 10 km-long window. Regarding the interval of the outcrops, the test results of sampling rates of more than 1.0\% with interpolation for data at non-sampled grids are not inferior to the full sample. Since the training data used in this study are computed on 5 m-spaced grids, extracting data from these grids with a 1.0\% probability is equivalent to conducting an inverse analysis from outcrop data that are distributed at 0.5 km intervals on average. Although the RMSEs of the model prediction certainly increase when the sampling rate decreases below 1.0 \%, the RMSE values does not drastically worsen until 0.5 \%. Therefore, even if the outcrop spacing is about 1 km, it should be possible to obtain a reasonable estimates of the flow characteristics.

These requirements for accurate inversion are attainable in the actual field. For example, \citet{Hirayama1977} correlated individual turbidites of the Pleistocene Otadai Formation distributed in the Boso Peninsula, Japan, on the basis of the key tuff beds. Their correlation covered a region over 30 km long with 33 outcrops. Thus, the average interval between outcrops was approximately 1 km. \citet{Amy2006} correlated individual beds in the Miocene Marnoso Arenacea Formation, Italy, using the Contessa MegaBed and an overlying ``columbine'' marker bed as the key beds. Their correlation covers 109 sections of approximately 30 m thick succession and extends over 120 km in a direction parallel to flow. Other studies in various regions (e.g., the Arnott Sandstone in France) also reported the correlation of individual turbidites in similar scale and frequency \citep{HESSE1974,Tokuhashi1979,Tokuhashi1989,Amy2000,Amy2004}. Furthermore, \citet{Bartolini1972} surveyed the Western Alboran Basin Plain, Mediterranean Sea, and discovered an individual turbidite on the sea floor at 49 cores over approximately 30 km. The records of cores in similar scale and intervals have also been reported by other studies of the modern submarine fans in different areas \citep{BORNHOLD1971, Pilkey1980}. In summary, although the method proposed in this study requires fairly high resolution data of turbidite individual beds correlated over a long distance, such conditions in ancient geological records as well as modern seafloor surveys can be achieved.

Besides these outcrop conditions, measurement errors in the field are another important factor for application. The test results suggest that the proposed inverse model of this study is very robust against random noise; random errors in the measured data have little effect on the results (Fig. \ref{fig:test_noise}). Therefore, even if localized and small-scale scouring and sedimentation occur due to some processes such as bottom currents after the deposition of a turbidite, results of inverse analysis will not be seriously affected. However, if deposits of multiple events are amalgamated to form a single thick massive sandstone, the hydraulic conditions reconstructed from the bed should be considerably different from the actual conditions. To avoid this situation, it is important to identify the erosional surface inside the bed carefully at the actual outcrop. In addition, it is safer not to analyze massive sandstones that are more than several meters thick, because they are likely to be amalgamated deposits.

Perhaps the most significant drawback to analyze actual turbidites is the assumption about the topography of the upstream submarine canyon. In this study, we tested doubling the length of the upstream slope and found that the predicted values for the concentration that were different from the original values (Fig. \ref{fig:test_slope_length}). In case of actual analysis, the upstream topography can be set correctly if the modern submarine fan is analyzed. Regarding ancient turbidites, however, some assumptions about the length and scale of the submarine canyons are necessary without measurements. In this case, it is recommended to set up various lengths of submarine canyons within a reasonable range, and to examine the degree to which these assumptions affect the inverse analysis results carefully. Nevertheless, it is worth noting that the test results were reasonable for velocity (Fig. \ref{fig:test_slope_length}), even if the assumption about the length of upstream slope was substantially different. This suggests that the inverse model proposed in this study can generally reconstruct the behavior of turbidity currents in sedimentary basins, even if the development process of turbidity currents upstream is different.


\subsection{Comparison with previous methodologies}
In existing inverse analysis methods of turbidity currents, the difference in depositional characteristics between the outputs of the forward model and the field observation is quantified as the objective function, and the initial and the boundary conditions of the forward model are determined by conducting optimization calculations to minimize the objective function \cite[e.g.,]{Nakao2017}. This is because models of turbidity currents are generally nonlinear and are difficult to linearize, especially when considering the entrainment of the basal sediment \citep{parker1986self}. Although the actual computational load depends on the choice of algorithm, this type of optimization calculation generally consists of multiple steps, and each step depends on the results of the previous calculation. Thus, the entire optimization procedure is difficult to parallelize. For instance, the kriging-based surrogate management method \citep{lesshafft2011towards} or the genetic algorithm \citep{Nakao2017} have been used to optimize the objective function for inversion of turbidity currents. In these methods, multiple calculations are conducted in each calculation step (generation), and the distribution of the objective function in the parametric space is iteratively estimated. Although the computations within each generation can be parallelized in this kind of algorithms, the next generation's computation depends on the results of the previous generation's computation, and therefore, the entire computation process cannot be parallelized. Thus, if the computational load of the forward model is high, the inverse analysis takes an unrealistic amount of time. 

\citet{Parkinson2017} applied the adjoint method with the gradient-based optimization algorithm. Although the differentiation of the layer-averaged model by the adjoint method greatly reduces the load of the gradient calculation, this approach still requires an iterative calculation for optimization. Thus, the sediment entrainment process is omitted from their model. Their model does not consider resuspension (entrainment) process of sediment, whereas suspended sand in turbidity currents is maintained by balancing the effects of particle settling and diffusion from the bottom (i.e. entrainment). Their model only considers advection and settling of particles, so that the suspended sediment quickly settles and be lost over short distances at realistic flow thicknesses and concentrations. The only way to transport large amounts of suspended sediment for long distance and to deposit thick turbidites without resuspension is to make the flow extremely thick or to suppose unusually high velocity or concentration. This is the reason for that the extremely thick flow depth (more than 3000 m) was obtained in their results. Their inversion method requires iterations that cannot be parallelized, so that the forward model needs to be simplified for this purpose. In addition, gradient-based optimization tends to have problems with initial value dependency and escaping from local optimal solutions. For this reason, the results of their inverse analysis of turbidites were quite unrealistic. In contrast, we were able to adopt "full model" that incorporate the entrainment process of suspended sand into our model without any problems. As a result, our inversion did not produce any anomalous reconstructions even though most of our test data exhibit thickness and grain size distributions similar to realistic turbidites. This strongly suggests the robustness of our inverse model and its applicability to real turbidites.

Another potential approach to optimization is the Markov Chain Monte Carlo (MCMC) method, but even with this method, repetition of the forward model calculation is unavoidable, since MCMC usually requires repetition of calculations of objective function, which cannot be parallelized, more than the order of $10^4$ time. The layer-averaged model of unsteady turbidity currents is probably not suitable for the forward models due to their computational load.

The approach proposed in this study is obviously superior to existing methods in terms of applicability to the field, as it allows computationally demanding models to be applied as forward models. The general relationship between the bed and the input parameters is learned by NN rather than adjusting the input parameters of the numerical model to reproduce the characteristics of specific individual beds. The objective function used in the training of this NN is not the difference between the features of the sediment, but the precision of the inverse analysis results themselves. The most computationally demanding part of the inverse analysis method proposed here is the generation of the training data for the NN. However, since the computations of the forward models are completely independent of each other, the generation of the training data can be conducted in parallel. Thus, our method enables us to easily prepare a large number of training data by using PC clusters, even for very computationally demanding forward models. In addition, the number of calculations required for training is not as high as other methods, specifically only approximately 3,000. It is also advantageous that the proposed method enables us to perform various tests for robustness or precision of inversion before application to field examples, because the NN outputs results of inverse analysis extremely fast. For these reasons, we consider that this study successfully generated an inverse model using the layer-averaged model for unsteady turbidity currents that can be applied to the field. 

\subsection{Limitations and future tasks}

The inverse model proposed in this study has several limitations. Inevitably, the accuracy of the inverse analysis is governed by the validity of the forward model that generates the training data. The present implementation of the inverse model uses the one dimensional layer-averaged model as the forward model, but this model is likely to be applicable only to sedimentary basins that are laterally constrained or to the inside of the submarine channels. The layer-averaged model of \citet{parker1986self} used in this study has been widely accepted, but various doubts have been recently raised such as the formulation of entrainment rates of basal sediment \citep{Dorrell2018} and ambient seawater \citep{Luchi2018}. The assumption of a lock exchange condition for the occurrence of turbidity currents may not be appropriate in some situations. 

Although \citet{Luchi2018} suggested that the a single layer model may not be sufficient for considering behavior of turbidity currents maintained over long distances, it is expected that such turbidity currents do not leave turbidites and create a bypassing zone. Otherwise, the concentration in the lower layers of turbidity currents decrease, and therefore the currents stop within a relatively short distance. Thus, a two-layer model of turbidity currents is not always necessary for inversion of bed-scale turbidites. However, modeling of continuos sustained turbidity currents is necessary for inverse analysis of the development of submarine fans and channel-levee systems in a larger scale. 

It is relatively easy to solve these problems described above. Without changing the framework of the proposed method, we can adapt to any situation by changing the forward model to generate the training data. For processes such as sediment transport, it is easy to revise the model to incorporate the state-of-the-art knowledge. By adopting computationally demanding models, inverse analysis using 2-D and 3-D forward models may be possible. In Future research, these issues should be addressed, and the methodology to actual field examples should be applied.

The analysis of ancient turbidites is an important issue in the future. However, even if ancient turbidites are analyzed, it is not possible to verify that the results obtained are correct, because the hydraulic conditions for ancient turbidity currents are unknown. Another way to verify the validity of the method is to reconstruct the hydraulic conditions of experimental turbidity currents from the turbidites deposited in the flume, and compare them with the measured values. The turbidity currents measured in the modern submarine canyons and their deposits would be another candidate to be used for the model verification.
