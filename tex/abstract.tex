%#!platex DeepLearningTurbidite.tex

  Although in situ measurements observed on modern frequently occurring turbidity currents have been performed, the flow characteristics of turbidity currents that occur only once every hundreds of years and deposit turbidites over a large area have not yet been elucidated. In this study, we propose a method for estimating the paleo-hydraulic conditions of turbidity currents from ancient turbidites by using machine learning. In this method, we hypothesize that turbidity currents result from suspended sediment clouds that flow down a steep slope in a submarine canyon and into a gently sloping basin plain. Using inverse modeling, we reconstruct seven model input parameters including the initial flow depth, the sediment concentration and the basin slope. A reasonable number (3,500) of repetition of numerical simulation using one-dimensional layer-averaged model under various input parameters generates a dataset of the characteristic features of turbidites. This artificial dataset is then used for supervised training of a deep learning neural network (NN) to produce an inverse model capable of estimating paleo-hydraulic conditions from data of the ancient turbidites. The performance of the inverse model is tested using independently generated datasets. Consequently, the NN successfully reconstructs the flow conditions of the test datasets. In addition, the proposed inverse model is quite robust to random errors in the input data. Judging from the results of subsampling tests, inversion of turbidity currents can be conducted if an individual turbidite can be correlated over 10 km at approximately 1 km intervals. These results suggest that the proposed method can sufficiently analyze field-scale turbidity currents.